
\documentclass{article}

\usepackage[margin=1.0in]{geometry}

\title{Functional Design Project Proposal}
\author{James Coleman Gibson}
\date{\today}

\begin{document}
\maketitle

\section*{Background}
Being a teaching assistant for CSSE304, Programming Language Concepts, is a very
large time commitment. The assistants for the class are asked to spend time in
the CS labs to assist students with concepts, debugging, are asked to give
feedback on assignments, and to grade exams. Examining students' coding
submissions is particularly time consuming, as the assistants must interpret
intentions and attempt to distinguish between small mistakes and fundamental
understanding. For assignments with efficiency requirements the assistants must
also determine control flow to determine whether things like \texttt{append}
will break linearity requirements in their specific position. In the beginning
of the course, we must give feedback to simplify expressions with things like
$\eta$ conversions and for shortening code with existing built in procedures. In
short, a huge amount of the assistants' time is spent working as a human code
analysis tool.

\section*{Proposal}
To simplify the lives of both the teaching assistants and the students in the
course, I propose the construction of a static analysis tool to be used by
graders and students alike. The feature list below is an approximate list of
features for such a tool.

\subsection*{Features}

\subsubsection*{\textit{Code Suggestions}}
A huge number of students in CSSE304 struggle with code simplification
suggestions, including reductions from \texttt{(append (list x) ...)} to
\texttt{(cons x ...)} and basic $\eta$-reductions. Implementing an
analysis tool to allow both these suggestions and others to be implemented would
prevent students from making these mistakes in the first place, leaving them
more time to learn more important topics and leaving the assistants more time to
give more meaningful feedback.

\subsubsection*{\textit{Procedure Arity Checking}}
One of the most frustrating errors to get in the interpreter project and the
individual assignments is the infamous \texttt{Incorrect number of arguments to
  \#<procedure>} error. This is difficult to debug for a variety of reasons. The
first and most basic is that the error does not include a line number of where
the procedure was called. This on its own is a rather simple issue which would
be easy to fix by using an interpreter which tracks source metadata. This could
be improved even further by including the source metadata in the procedure
itself, allowing you to also see where the offending procedure was defined. Even
better would be if you didn't have to run the code at all to detect these errors.
Using a $k-CFA$ implementation to find arity violations would save students from
needing to sort through pages of anonymous procedure traces and would be a
stepping stone into other useful analyses.

\subsubsection*{\textit{HTTP Interface}}
For this project I will be using Haskell, which is not a commonly used language
by either students or professors at Rose-Hulman. As such, installation becomes a
tricky issue. Allowing for expression based or program based analysis through an
HTTP interface would make this much more simple, as only one server in the
school would actually need to be running Haskell code.

\subsubsection*{\textit{An Interpreter}}
Although Chez Scheme does feature a number of debugging tools, many of them can
be difficult to use for beginner Scheme programmers. This would not such a
significant issue, except that Chez Scheme's error messages are uninformative as
to the locations of errors. An interpreter built specifically to be used for
debugging would allow students to spend more time learning the language, and
less time learning to use \texttt{trace}.

\subsubsection*{\textit{Shape Analysis}}
Several of the more difficult assignments to give feedback on in CSSE304 are the
programs with specific requirements on time and space complexity. Detecting this
automatically would save many hours of time for assistants in the course and
would help students to learn why their code does not meet these requirements
before they actually lose points. A potential way to solve this is through
program shape analysis, to detect properties like whether there are multiple
visits to a node in a tree.

\subsubsection*{\textit{Advanced Testing}}
A last possibility is to implement a more advanced testing system for the
course. This could be done with something like test generation or through
property based testing. Although this is dependent on the interpreter feature
for the project, this has a great potential to help prevent incorrect solutions
from slipping by the grading server.

\end{document}
